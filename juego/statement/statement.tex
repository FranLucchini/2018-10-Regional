\documentclass{oci}
\usepackage[utf8]{inputenc}
\usepackage{lipsum}

\title{Ejemplo}

\begin{document}
\begin{problemDescription}
Comcap decidió gastarse sus últimos ahorros aprendiendo a programar, pues es una habilidad increiblemente útil para la vida. Tristemente se dio cuenta de que no le queda un peso para agregar su nueva habilidad a su curriculum y no lo aceptan en ningún trabajo sin ella. Por esto, Comcap decidió visitar el casino clandestino de la ciudad, donde puedes jugarte hasta tu vida.

Ya en el casino, Comcap determinó que había un solo juego en el que puedes ganar sin tener suerte, pero se debe tener un poder de procesamiento de información enorme para ganar. 

El juego se llama $Mega oci X$
La idea es un juego donde se tiene un pasillo de largo n. Se tiene un personaje (el jugador) que comienza en una posición $x$ de la matriz. También hay $k$ lasers pegados a las paredes del pasillo con posiciones $a_i$, con $0\leq i<k$. Los lasers vienen de a pares, uno en cada lado del pasillo, y disparan un rayo recto.

La idea es que los lasers se desplazan en el pasillo de forma continua. Cada laser tiene un entero $m_i$ que indica la cantidad de bloques que avanza cada segundo, y un entero $c_i$,1 o -1, que indica la orientación del laser, es decir, si avanza en la matriz o retrocede respectivamente. También, cada laser dispara cada segundo \textbf{ENTERO} (en el segundo 0 no disparan). Agregaré que los lasers $desaparecen$ una vez que salen de la matriz.

El personaje también tiene un entero $L$ que indica la cantidad de bloques que avanza cada segundo. El objetivo es que el personaje llegue al extremo derecho del pasillo antes de morir por un laser. Dadas todas las variables mencionadas, se debe indicar si existe una forma en que el jugador llegue al extremo derecho.

Solución: Vemos que por cada laser, el punto de intersección es único, hay infinitos puntos de intersección o no existe intersección. Luego, si es único, satisface la ecuación $sL+x=a_i+c_im_i(s-1)$, podemos encontrar que (segundos) debe ser entero, $s>=0$ y $s<=segundos$ $totales$ $para$ $llegar$. Además de esto, se debe verificar la situación inicial dado que los lasers parten no disparando. 

Creo que hay varias opciones para las sub-tareas. Siempre se cumple que $1<=m_i, L, n<=1e15$
\end{problemDescription}

\begin{inputDescription}
La primera fila contiene 4 enteros $n, x, L, k$ $-$ la cantidad de filas de la matriz, la cantidad de columnas de la matriz, la fila en la que se encuentra el jugador, la columna en la que se encuentra el jugador, la cantidad de bloques que avanza el jugador en un segundo y la cantidad de lasers totales en juego.\\
Luego siguen $k$ filas con 3 enteros $a_i$, $m_i$ y $c_i$, que indican la posición en que se encuentra el laser $i-ésimo$, los bloques que se mueve cada segundo y su orientación respectivamente.
\end{inputDescription}

\begin{outputDescription}
La salida consiste en una palabra, $SI$ o $NO$, dependiendo de si el jugador ganará el juego o no.
\end{outputDescription}

\begin{scoreDescription}
  \score{20} $c_i=1$, $m_i=2, L=1$, $1\leq n \leq 10^{15}$, $0\leq k\leq 10^5$
  \score{20} $c_i=-1$, $m_i=1, L=1$, $1\leq n \leq 10^{15}$, $0\leq k\leq 10^5$
  \score{30} $1\leq m_i, L, n\leq 10^{6}$, $0\leq k\leq 50$
  \score{30} $1\leq m_i, L, n\leq 10^{15}$, $0\leq k\leq 10^5$
\end{scoreDescription}

\begin{sampleDescription}
\sampleIO{sample-1}
\sampleIO{sample-2}
\end{sampleDescription}

\end{document}
