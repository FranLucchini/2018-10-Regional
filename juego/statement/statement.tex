\documentclass{oci}
\usepackage[utf8]{inputenc}
\usepackage{lipsum}

\title{Ejemplo}

\begin{document}
\begin{problemDescription}
La idea es un juego donde se tiene una matriz de dimensiones (n, m). Se tiene un personaje (el jugador) que comienza en una posición $(x, y)$ de la matriz. También hay $k$ lasers con posiciones $(a_i, b_i)$ en la matriz, con $0\leq i<k$. los lasers disparan como los movimientos de la torre en ajedrez, es decir, disparan a lo largo de su columna y fila.

La idea es que los lasers se desplazan horizontalmente en la matriz de forma continua. Cada laser tiene un entero $m_i$ que indica la cantidad de bloques que avanza cada segundo, y un entero $c_i$,1 o -1, que indica la orientación del laser, es decir, si avanza en la matriz o retrocede respectivamente. También, cada laser dispara cada segundo \textbf{ENTERO} (en el segundo 0 no disparan). Agregaré que los lasers $desaparecen$ una vez que salen de la matriz.

El personaje también tiene un entero $L$ que indica la cantidad de bloques que avanza cada segundo (horizontalmente). El objetivo es que el personaje llegue al extremo horizontal derecho de la matriz antes de morir por un laser. Dadas todas las variables mencionadas, se debe indicar si existe una forma en que el jugador llegue al extremo horizontal derecho.

Solución: Vemos que por cada laser, el punto de intersección es único, hay infinitos puntos de intersección o no existe intersección. Luego, si es único, satisface la ecuación $sL+y=b_i+c_im_i(s-1)$, podemos encontrar que (segundos) debe ser entero, $s>=0$ y $s<=segundos$ $totales$ $para$ $llegar$. Además de esto, se debe verificar la situación inicial dado que los lasers parten no disparando. 

Creo que hay varias opciones para las sub-tareas. Siempre se cumple que $0<=n<=50$, $1<=m_i, L, m<=1e15$
\end{problemDescription}

\begin{inputDescription}
La primera fila contiene 6 enteros $n, m, x, y, L, k$ $-$ la cantidad de filas de la matriz, la cantidad de columnas de la matriz, la fila en la que se encuentra el jugador, la columna en la que se encuentra el jugador, la cantidad de bloques que avanza el jugador en un segundo y la cantidad de lasers totales en juego.\\
Luego siguen $k$ filas con 4 enteros $a_i$, $b_i$, $m_i$ y $c_i$, que indican la fila y columna en que se encuentra el laser $i-ésimo$, los bloques que se mueve cada segundo y su orientación respectivamente.
\end{inputDescription}

\begin{outputDescription}
La salida consiste en una palabra, $SI$ o $NO$, dependiendo de si el jugador ganará el juego o no.
\end{outputDescription}

\begin{scoreDescription}
  \score{20} $c_i=-1$, $m_i, L=1$, $1\leq m \leq 10^{15}$, $0\leq k\leq 10^5$
  \score{20} $m_i=1$, $L=1$, $1\leq m \leq 10^{15}$, $0\leq k\leq 10^5$
  \score{30} $1\leq m_i, L, m\leq 10^{6}$, $0\leq k\leq 50$
  \score{30} $1\leq m_i, L, m\leq 10^{15}$, $0\leq k\leq 10^5$
\end{scoreDescription}
Siempre se cumple que: 
$0\leq n\leq 50$\\

\begin{sampleDescription}
\sampleIO{sample-1}
\sampleIO{sample-2}
\end{sampleDescription}

\end{document}
