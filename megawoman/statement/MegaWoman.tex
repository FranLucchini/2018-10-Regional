\documentclass{oci}
\usepackage[utf8]{inputenc}
\usepackage{lipsum}
\usepackage{tikz}
\usetikzlibrary{calc}
\usepackage{tabularx}

\title{Mega Woman}



\begin{document}
\begin{problemDescription}
\end{problemDescription}

\begin{scoreDescription}
  \score{20} Se probarán varios casos donde la velocidad de Mega Woman es 1,
  todos los robots se mueven hacia la derecha con velocidad 2 y tanto la
  cantidad de casillas como de robots es menor o igual que $10^5$ ($v_r=2, v_w=1,
  R\leq 10^5, N\leq 10^5$).

  Para comenzar a resolver el problema, lo primero que debemos considerar es que
  tanto los movimientos de MegaWoman como de los robots se pueden ver como funciones
  que dependen del tiempo. Luego, para saber si alg\'un robot ataca a MegaWoman,
  debemos calcular la interseci\'on entre ambas funciones. Si ese tiempo es positivo
  y ocurre antes de llegar a la casilla final, se considerar\'a que MegaWoman no 
  logra escapar en esa situaci\'on.

  \begin{enumerate}
    \item \textbf{Funci\'on de movimiento de MegaWoman}: sea $P_w$ la posici\'on
    inicial de MegaWoman. Dado eso, la funci\'on
    de distancia ser\'ia:
    $$f_w(t) = P_w + t$$
    \item \textbf{Funci\'on de movimiento de un robot}: sea $P_r$ la posici\'on
    inicial del robot. Dado eso, la funci\'on
    de distancia ser\'ia:
    $$f_r(t) = P_r + 2 \cdot t$$
    En este caso, multiplicamos por $2$, porque el robot avanza de dos en dos hacia la
    derecha.
  \end{enumerate}

  \textbf{Encontrar la intersecci\'on}: la intersecci\'on entre dos funciones se
  encuentra cuando ambas funciones tienen el mismo valor, o sea cuando 
  $f_w(t) = f_r(t)$ en nuestro caso. Con esta igualdad, podemos despejar el valor
  de $t$ en el que se intersectan (esto significa que el robot atacar\'ia a MegaWoman 
  con una bomba en ese momento). A continuaci\'on se muestra c\'omo queda expresado $t$:
  $$ f_w(t) = f_r(t) $$
  $$ P_w + t = P_r + 2 \cdot t $$
  $$ P_w - P_r =  1 \cdot t $$
  $$ P_w - P_r =  t $$

  Es muy importante asegurarse de que:
  \begin{enumerate}
    \item El tiempo que obtenemos no sea negativo: $0 <= t$
    \item La posici\'on en la que ocurre el ataque (la llamaremos $dist$) no sea 
    mayor al\\ n\'umero de casillas ($N$): $dist < N $ 
  \end{enumerate}

  Si se cumplen todas estas condiciones, debemos imprimir \texttt{NO}. 
  En caso contrario, imprimimos \texttt{SI}.

  \textbf{Encontrar la posici\'on  en que ocurre un ataque}: si ya tenemos el valor
  de $t$, calcular la posici\'on del ataque es simplementa reemplazar el valor de $t$
  en alguna de las funciones. En este caso, elegimos $f_w(t)$ y supongamos que $t = 2$
  y que $P_w = 2$:
  $$dist = f_w(2)$$
  $$dist = P_w + 2$$
  $$dist = 2 + 2 = 4$$

  \textbf{Casos especiales}:
  \begin{enumerate}
    \item Si la posici\'on inicial de MegaWoman es igual a la de un robot ($P_w = P_r$), imprimimos \texttt{NO}.
    \item Si la posici\'on inicial de un robot es mayor que la de MegaWoman ($P_w < P_r$), sabemos
    que no tendr\'an intersecci\'on y nos podemos saltar a ese robot. Esto es as\'i
    porque los robots avanzan a mayor o igual velocidad que MegaWoman y en la misma 
    direcci\'on, por ende, llegar\'a antes al final de las casillas.
  \end{enumerate}

  \noindent\makebox[\linewidth]{\rule{\paperwidth}{0.2pt}}

  \score{20} Se probarán varios casos donde la velocidad de Mega Woman es 1,
  todos los robots se mueven hacia la izquierda con velocidad -1 y tanto la cantidad
  de casillas como de robots es menor que $10^5$ ($v_r=-1, v_w=1, R\leq 10^5,
  N\leq 10^5$).

  En esta subtarea, haremos algo similar pero, cambiaremos un poco la funci\'on
  de movimiento del robot.

  \begin{enumerate}
    \item \textbf{Funci\'on de movimiento de MegaWoman}: sea $P_w$ la posici\'on
    inicial de MegaWoman. Dado eso, la funci\'on
    de distancia ser\'ia:
    $$f_w(t) = P_w + t$$
    \item \textbf{Funci\'on de movimiento de un robot}: sea $P_r$ la posici\'on
    inicial del robot. Dado eso, la funci\'on
    de distancia ser\'ia:
    $$f_r(t) = P_r - t$$
    En este caso, multiplicamos por $-1$, porque el robot avanza de a una casilla 
    hacia la izquierda.
  \end{enumerate}

  \textbf{Encontrar la intersecci\'on}:
  $$ f_w(t) = f_r(t)$$
  $$ P_w + t = P_r - t $$
  $$ P_w - P_r =  -2 \cdot t $$
  $$ \frac{P_w - P_r}{-2} =  t $$
  $$ \frac{P_r - P_w}{2} =  t $$

  \newpage

  Ahora que hay una divisi\'on por un n\'umero distinto de $1$, debemos asegurarnos
  tambi\'en de que el valor del tiempo sea un n\'umero entero. Lo a\~nadimos a la lista:
  \begin{enumerate}
    \item El tiempo que obtenemos no sea negativo: $0 <= t$
    \item La posici\'on en la que ocurre el ataque no sea 
    mayor al n\'umero de casillas ($N$): $dist < N $
    \item El tiempo es un n\'umero entero: $ (P_r - P_w) \% 2 = 0$
  \end{enumerate}

  Si se cumplen todas estas condiciones, debemos imprimir \texttt{NO}. 
  En caso contrario, imprimimos \texttt{SI}.

  \textbf{Casos especiales}:
  \begin{enumerate}
    \item Si la posici\'on inicial de MegaWoman es igual a la de un robot ($P_w = P_r$), imprimimos \texttt{NO}.
    \item Si la posici\'on inicial de un robot es menor que la de MegaWoman ($P_w > P_r$), sabemos
    que no tendr\'an intersecci\'on y nos podemos saltar a ese robot. Esto ocurre
    porque los robots avanzan en posici\'on opuesta a la de MegaWoman y por cada segundo
    que pase se alejar\'an m\'as, por lo que no se van a encontrar.
  \end{enumerate}

  \noindent\makebox[\linewidth]{\rule{\paperwidth}{0.2pt}}

  \score{25} Se probarán varios casos en que el número de robots es menor que
  50, el número de casillas es menor o igual que $10^5$ y sin restricciones sobre las
  velocidades ($R\leq 50, N\leq 10^5$).

  En esta subtarea, ya no hay restricci\'on sobre los valores de las velocidades. 
  Ahora es necesario usar los valores de $v_w$ (velocidad de MegaWoman) y $v_r$ (velocidad
  de un robot).

  \begin{enumerate}
    \item \textbf{Funci\'on de movimiento de MegaWoman}: sea $P_w$ la posici\'on
    inicial de MegaWoman y $v_w$ su velocidad. Dado eso, la funci\'on
    de distancia ser\'ia:
    $$f_w(t) = P_w + v_w \cdot t$$
    \item \textbf{Funci\'on de movimiento de un robot}: sea $P_r$ la posici\'on
    inicial del robot y $v_r$ su velocidad. Dado eso, la funci\'on
    de distancia ser\'ia:
    $$f_r(t) = P_r + v_r \cdot t$$
  \end{enumerate}

  \textbf{Encontrar la intersecci\'on}:
  $$ f_w(t) = f_r(t) $$
  $$ P_w + v_w \cdot t = P_r - v_r \cdot t $$
  $$ P_w - P_r =  v_r \cdot t - v_w \cdot t $$
  $$ P_w - P_r =  t \cdot (v_r - v_w) $$
  $$ \frac{P_r - P_w}{v_r - v_w} =  t $$

  Seguimos considerando las mismas condiciones que antes:
  \begin{enumerate}
    \item El tiempo que obtenemos no sea negativo: $0 <= t$
    \item La posici\'on en la que ocurre el ataque no sea 
    mayor al n\'umero de casillas ($N$): $dist < N $
    \item El tiempo es un n\'umero entero: $ (P_w - P_r) \% 2 = 0$
  \end{enumerate}

  Si se cumplen todas estas condiciones, debemos imprimir \texttt{NO}. 
  En caso contrario, imprimimos \texttt{SI}.

  \textbf{Encontrar la posici\'on  en que ocurre un ataque}: si ya tenemos el valor
  de $t$, calcular la posici\'on del ataque es simplementa reemplazar el valor de $t$
  en alguna de las funciones. En este caso, elegimos $f_w(t)$ y supongamos que $t = 2$,
  $P_w = 2$ y que $v_w = 3$:
  $$dist = f_w(2)$$
  $$dist = P_w + v_r \cdot 2$$
  $$dist = 2 + 3 \cdot 2 = 8$$

  \textbf{Casos especiales}:
  \begin{enumerate}
    \item Si la posici\'on inicial de MegaWoman es igual a la de un robot ($P_w = P_r$), imprimimos \texttt{NO}.
    \item Si la posici\'on inicial de un robot es menor que la de MegaWoman ($P_w > P_r$) y su
    velocidad es negativa ($v_r < 0$): no se encuentra con MegaWoman; nos saltamos este robot.
    \item Si la posici\'on inicial de un robot es mayor que la de MegaWoman ($P_w > P_r$) y su
    velocidad es positiva ($v_r > 0$): no se encuentra con MegaWoman; nos saltamos este robot.
  \end{enumerate}

  \noindent\makebox[\linewidth]{\rule{\paperwidth}{0.2pt}}

  \score{35} Se probarán varios donde el número de robots es menor o igual que
  $10^5$, la cantidad de casillas es menor o igual que $10^{12}$ y sin
  restricciones sobre las velocidades ($R\leq 10^5, N \leq10^{12}$).

  En este \'ultimo caso, debemos considerar que los valores posibles para el n\'umero de
  casillas es mayor que la capacidad de un \texttt{int}, por lo que pasamos los tipos
  de nuestras variables a \texttt{long}.
\end{scoreDescription}

\end{document}