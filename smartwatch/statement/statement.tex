\documentclass{oci}
\usepackage[utf8]{inputenc}
\usepackage{lipsum}

\title{Reloj inteligente}

\begin{document}
\begin{problemDescription}
  Hace poco Nicolás se compró un \emph{reloj inteligente}.
  Ahora cada vez que realiza una actividad deportiva, Nicolás lleva su nuevo
  reloj pues este le entrega en tiempo real información muy útil como su
  velocidad o ritmo cardíaco.
  Por otro lado, su reloj se conecta también con una aplicación en su celular y
  lleva un registro de todas las actividades que realiza.
  La parte entretenida es que la aplicación propone ciertos desafíos y le otorga
  un premio cada vez que rompe algún récord en alguno de ellos.
  En uno de los desafíos Nicolás puede escoger un entero $M$ y entre todas sus
  actividades la aplicación le dirá el tiempo mínimo en que ha podido recorrer
  $M$ metros.

  Después de un tiempo ocupando la aplicación, Nicolás se ha dado cuenta que el
  desafío no funciona como él esperaba.
  Cada vez que empieza una actividad, el reloj guarda el tiempo cada $M$
  metros recorridos.
  Por ejemplo, si $M=3$ la aplicación guardará el tiempo en
  recorrer los primero 3 metros, luego el tiempo entre el metro 3 y
  el 6, entre el 6 y el 9, etc.
  A Nicolás le gustaría que la aplicación considerara todos los intervalos en
  los que recorrió $M$ metros.
  Por ejemplo, para $M=3$ otros intervalos posibles podrían ser entre el
  metro 2 y 5 o entre el 7 y el 10.

  Para la suerte de Nicolás el reloj es realmente inteligente y puede
  programarse para añadir nuevas funcionalidades.
  Lamentablemente, Nicolás es realmente malo programando y necesita de tu ayuda.

  Internamente, el reloj toma una \emph{muestra} cada segundo y guarda la
  distancia en metros que fue recorrida durante ese segundo.
  Un intervalo corresponde a un conjunto contiguo de muestras.
  Como es posible que los intervalos no sumen exactamente $M$ metros, a
  Nicolás le interesan los intervalos que sean lo más cercanos posible a $M$
  metros.
  Un intervalo \emph{válido} es uno cuya distancia es mayor o igual que $M$ metros,
  tal que al quitarle la última muestra la distancia queda menor que $M$ metros.
  Dada la descripción de las muestras tu tarea es determinar el intervalo
  válido de menor tiempo.

\end{problemDescription}

\begin{inputDescription}
  La primera línea de la entrada contiene dos enteros $M$ ($0<M\leq 10^5$) y $N$
  ($N > 0$) correspondientes respectivamente al valor que Nicolás ha escogido
  para el desafío y la cantidad de muestras.
  La siguiente línea contiene $N$ enteros correspondientes a la cantidad de
  metros en cada intervalo.
  Todos los enteros serán mayores o iguales que 0 y menores que $10^9$
\end{inputDescription}

\begin{outputDescription}
  La salida debe contener un único entero correspondiente al tiempo en segundos
  del intervalo válido de menor tiempo.
\end{outputDescription}

\begin{scoreDescription}
  \score{20} Se probarán varios casos donde $N<=10^2$
  \score{30} Se probarán varios casos donde $N<=10^3$
  \score{50} Se probarán varios casos donde $N<=10^6$
\end{scoreDescription}

\begin{sampleDescription}
\sampleIO{sample-1}
\sampleIO{sample-2}
\end{sampleDescription}

\end{document}
