\documentclass{oci}
\usepackage[utf8]{inputenc}
\usepackage{lipsum}
\usepackage{amsmath,amsthm,amsfonts, amssymb}
\usepackage{tikz}

\title{Las sumas de Gabriela}

\begin{document}
\begin{problemDescription}
Gabriela necesita ahorrar mucho dinero.
Es por esto que pasa todo el día haciendo cálculos para saber cuanto dinero
ha gastado.
Las cálculos de Gabriela son simples, dada una lista de números $a_1, a_2,
\dots, a_n$ debe sumarlos para obtener un valor $S$.

Gabriela mantiene en su libreta un registro de todas las sumas que ha hecho,
pero ella no quiere que la gente sepa cuanto dinero ha gastado.
Es por esto que diseñó una estrategia para poder ocultarlo.
Su estrategia consiste en anotar los números en sus sumas sin escribir
ningún signo o espacio de separación.
Por ejemplo, para la suma $1111+2222+3333=6666$ ella escribiría el siguiente
registro en su libreta $1111222233336666$.

Dado un registro, a Gabriela le interesa poder recuperar la suma original.
Preocupada de no poder hacerlo, decidió agregar información extra incluyendo al
principio de todos sus registros la cantidad de sumandos en la operación.
Para el ejemplo anterior la cantidad de sumandos es $3$ y por lo tanto el
registro final que Gabriela escribiría en su libreta es $31111222233336666$.

Por casualidad acabas de recibir en tus manos la libreta de Gabriela.
?`Crees ser capaz de descifrar sus cálculos?
\end{problemDescription}

\begin{inputDescription}
La entrada consiste en una línea conteniendo una cadena de largo $M$ formada
únicamente por los siguientes caracteres $\{1,2,3,4,5,6,7,8,9\}$ (no contiene el
$0$).
Esta cadena corresponde a uno de los registros de Gabriela, es decir, para
alguna lista de números $a_1,a_2\dots a_n$ cuya suma es $S$, la línea tendrá el
formato:
$$n a_1 a_2 \dots a_nS$$
Se garantiza que la suma tiene al menos dos sumandos ($n\geq 2$) y que todos los
números son mayores que 0 y menores o iguales que $10^9$.
\end{inputDescription}

\begin{outputDescription}
La salida debe contener el registro de Gabriela agregando espacios 
entremedio de los números, es decir, debe tener el formato:
$$n\textvisiblespace a_1\textvisiblespace a_2\textvisiblespace
\dots\textvisiblespace a_n\textvisiblespace S$$
Está garantizado que siempre existe una forma válida de separar el registro y que
esta es única.
\end{outputDescription}

\begin{scoreDescription}
  \score{40} Se probarán varios casos donde $M\leq 15$.
  \score{60} Se probarán varios casos donde $M\leq 28$.
\end{scoreDescription}

\begin{sampleDescription}
\sampleIO{sample-1}
\sampleIO{sample-2}
\end{sampleDescription}

\end{document}
