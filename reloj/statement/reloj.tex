\documentclass{oci}
\usepackage[utf8]{inputenc}
\usepackage{lipsum}
\usepackage{tikz}

\title{Reloj inteligente}

\begin{document}

\begin{scoreDescription}
  \score{20} Se probarán varios casos donde $N\leq 10^2$
  \score{30} Se probarán varios casos donde $N\leq 2\times 10^3$

  El algoritmo que vamos a seguir recorre en un bucle el arreglo de
  intervalos y para cada posici\'on, revisar\'a los valores a la derecha en 
  un segundo bucle. El primer bucle representa la posic\'on inicial del intervalo
  y el segundo, la posici\'on final del intervalo.
  
  El onjetivo es dada una posici\'on inicial, avanzar la posici\'on final hasta
  lograr sumar una cantidad de metros igual o mayor al valor de $M$.
  Para encontrar el tiempo que ha transcurrido, debemos restar la posici\'on inicial a 
  la posici\'on final. Por ejemplo, si tenemos que estamos
  revisando la posici\'on $3$ y avanzamos hasta la $7$ en el segundo bucle, el
  tiempo total es $7 - 3 = 4$.

  Cuando hemos obtenido una suma mayor o igual a $M$, debemos preguntarnos si
  la soluci\'on actual es mejor que una encontrada anteriormente. Para eso,
  debemos mantener una variable que inicialmente tendr\'a el mayor valor posible
  y la actualizaremos a medida que encontremos un tiempo menor.

  \score{50} Se probarán varios casos donde $N\leq 10^6$

  Como en esta subtarea trabajamos con n\'umeros m\'as grandes, pensamos en
  una forma de optimizar el c\'odigo que tenemos.

  En la soluci\'on hasta ahora, cada vez que cambiamos de posici\'on inicial,
  reiniciamos el valor de la suma a cero y volvemos a buscar un valor para la
  posici\'on final. El problema es que estamos repitiendo operaciones muchas
  veces.

  Supongamos que $M = 8$, $N=7$ y el arreglo es $A = [1,3,2,1,2,4,3]$.

  Cuando la posici\'on inicial es $0$, avanzamos hasta la posici\'on final $4$
  para poder igual o superar $M$, o sea, hacemos: $1+3+2+1+2 = 9$. Luego,
  cambiamos a la posici\'on inicial en $1$, avanzamos tambi\'en hasta $4$, haci\'endo
  la suma $3+2+1+2 = 8$. Notamos que en ambos casos sumamos las posiciones
  de $A[1]$ hasta $A[4]$.

  Para evitar tener que hacer sumas innecesarias, lo ideal ser\'ia mantener la
  posici\'on final fija y s\'olo moverla en caso de ser necesario. Esto significa
  que s\'olo moveremos la posici\'on final si es que al mover la posici\'on
  inicial, la suma que nos queda en menor a $M$.

  Dado eso, ya no tenemos que redefinir el valor de la suma cuando cambiamos
  la posici\'on inicial. S\'olo debemos asegurarnos de restar los metros
  de la posici\'on inicial anterior para no inlcuir ese valor en la suma con el nuevo
  valor.

  Por ejemplo, siguiendo el ejemplo anterior, si ahora la posici\'on inicial es 
  $1$, debemos restar el valor de $A[0]$ a la suma para tener el valor correcto.
  Como la suma sigue siento mayor o igual a $M$, no movemos la posici\'on final.

  Luego, cuando queremos cambiar la posici\'on inicial de $1$ a $2$, primero restamos
  $A[1]$ de la suma: $suma = 8 - A[1] = 8 - 3 = 5$. Como la suma es menor a $M=8$, aumentamos 
  la posici\'on final de $4$ a $5$ y obtenemos: $suma = 5 + A[5] = 5 + 4 = 9$. Hemos
  logrado una suma mayor a $M$.

\end{scoreDescription}

\end{document}
