\documentclass{oci}
\usepackage[utf8]{inputenc}
\usepackage{lipsum}
\usepackage{amsmath,amsthm,amsfonts, amssymb}
\usepackage{tikz}

\title{Whitespace}

\begin{document}
\begin{problemDescription}
Los organizadores de las olimpiadas de informática adoran hacer sufrir a los competidores. Para variar esta no es la excepción. Había una vez una hoja de cálculo donde cada línea seguía el formato:

$$n\ a_1\ a_2\ \dots\ a_n\ suma$$

es decir, cada línea comenzaba con un entero positivo $n$, luego le seguían $n$ enteros $a_1, a_2, \dots a_n$ y finalmente un entero $suma$ correspondiente a la suma de los $n$ números. Originalmente todos estos números estaban separados entre sí por espacios en blanco. Uno de los organizadores pensó que sería muy divertido remover todos los espacios en blanco y encargarles a los desdichados competidores recuperarlos. Obviamente los demás organizadores estuvieron de acuerdo. ¿Te animas a hacerlo?
\end{problemDescription}

\begin{inputDescription}
La entrada consiste un único string de dígitos decimales siguiendo el formato:
$$n a_1 a_2 \dots a_n suma$$
es decir, una línea de la hoja de cáculo original pero con los espacios en blanco removidos. Todos los números son enteros, n siempre es positivo y ningún número está escrito con ceros a la izquierda y ningún número es mayor a $10^{18}$.
\end{inputDescription}

\begin{outputDescription}
La salida debe seguir el formato:
$$n\ a_1\ a_2\ \dots\ a_n\ suma$$
es decir, la línea original con los espacios en blanco recuperados. Está garantizado que sólo existe una única forma válida de separar los números.
\end{outputDescription}

\begin{scoreDescription}
  \score{40} largo del string $<= 15$
  \score{60} largo del string $<= 28$
\end{scoreDescription}

\begin{sampleDescription}
\sampleIO{sample-1}
\sampleIO{sample-2}
\end{sampleDescription}

\end{document}
